\chapter{Information Flow Tracking Notes}

Security policies and security mechanisms. These are two of the main concepts covered in this course
and we will delve into it a little bit more in this assignment.

In the Policy lecture, we discussed how we can model security policies as well as some of the more
famous security policies such as BLP and BIBA. In any of these policies, we stated that the crux of the
problem is trying to figure out and define what Subjects has access to what Objects and in what Way.
One question that you might ask is, okay, let’s say we have a policy - how do we enforce it? One way is
through information flow tracking.

Imagine that you are a security professional at an organization. You know that someone has been
stealing the company’s IP and would like a solution. What can you do? Now before we start to throw
ideas out there, perhaps we might want to consider how this is done in the real world. Instead of
thinking about the theft of information, perhaps we can think about the theft of merchandise. This is
known as ``loss prevention.''

We probably have lots of experience with loss prevention technologies such as the little tags that are
attached to clothing. The main idea is to tag every single article of clothing (ones that are worth a lot).
When someone purchases the article, the cashier will use a special machine to remove the tag. If on the
other hand the tag was not removed (e.g., you did not pay for it) then sensors at the entrance/exit will
pick up the tag and sound the alarm.

This seems to work well enough, so let’s try to dig down into the fundamentals as we like to do in this
course. If we think about this particular mechanism, we will see that it involves a ``tag'' that is used to tag
articles of clothing. In addition to the tag, there must also be a number of ``sensors'' that can be used to
identify the presence of tags. Finally, there is a machine or technique that can be used to ``remove'' the
tags. This brings us to four different parts, a {\em tag}, a machine to {\em attach} the tag in the first place, a {\em sensor},
to identify the tag, and a machine to {\em manipulate} (remove) the tag.

Let’s try to apply the same idea to the computer world. In the computer realm, instead of an article of
clothing, we have an article of information -- a piece of data. So instead of tagging clothing, we now
have to tag data.

Now one of the main observations that we made during the first module was that computers are
different from the real world because of the ability to make exact copies of data. So, imagine if you were
in a store and was able to take an article of clothing and make an exact copy of it. What should we do
with that tag? Should the copy also have a tag associated with it or should the copy not have a tag?

In the end, it would only make sense if the new copy also has a security tag. Therefore, when we make a
copy of digital information, we must also make a copy of the tag. The problem is more nuanced though.
In a computer, we don’t only have to make copies of information, but we can also transform it. For
example, what happens if we compress a file. Should the compressed file also contain a tag? What we
encrypt it? What if we take every single byte and add 1 to it? The list continues.

If you are not convinced with those examples, what if you add 1 to a piece of numerical data and then
followed by a subtract 1? Mathematically it would mean $data + 1 - 1$ which is the same thing as $data$. In other
words, since there are transformations that are equivalent to a copy, we need to worry more than just
the copy operations.

So, due to the ability to readily transform any data (especially in a way that can be reversed like the add,
sub example) we not only need to tag copies of the data, but also any transformations of the data. This
means that we now have an additional requirement to propagate the tag across different operations on
digital data.

Given all of this, we now have 4 different parts: {\em tags}, {\em sources} (the thing to attach the original tag), {\em sinks}
(the equivalent of the sensor at the entrance/exit), and {\em propagation rules} (including all manipulations
and not just the attachment and removal). These parts constitute our ``model'' of the problem.

This assignment is designed to give you some insights into how difficult this is in practice. The first part
of the assignment asks you to look into the details on how one might go about figuring out how data is
tagged and how the tags are propagated through operations. To be clear, the operations that you are
asked to look into are very basic operations that might be part of a CPU’s instruction set architecture.

While we would prefer to define the propagation rules (rules that determine how tags are propagated)
on large operations such as copying, encrypting, compressing, etc. we can’t do this because it is difficult,
if not impossible, to identify all possible copying, encrypting, compressing, and other functions in a
computer. When we think we have enumerated all of them, an attacker can simply implement a new
function or operation.

The one thing that an attacker cannot change is the CPU that is used to execute the instructions that
make up these functions. Therefore, it makes sense to figure out how to propagate tags using these
basic instructions -- we will call the decision to apply propagation rules to individual instructions, blocks,
functions, libraries, programs, or others the {\em analysis granularity}. These describe at what granular level we are analyzing the tags.

Let’s go back to our clothing example for a second. This time, instead of just tagging clothing to prevent
theft, imagine if you were also worried about the intellectual property - the design for example. Perhaps
you can imagine working for a company that has a new top-secret thing that is being showcased to some
visitors. You want your visitors to be able to see the new fangled device, but you don’t want them to be
able to steal it (therefore the tagging) nor do you want them to be able to take a picture of it because
you are worried that they might be able to reverse engineer the device from the photo.

In this case, taking a picture is just an example of some of these more advanced functions similar to
encryption in a computer. What do we do? What if we really want the visitors to take pictures all they
want, as long as they don’t take a picture of the specific thing? (If any of you have visited the Crown
Jewels, then you will know that visitors can take pictures of everything else except the Crown Jewels (see
\url{https://www.huffingtonpost.com/2014/06/12/places-you-cant-take-photos_n_5411226.html} for some
examples).

EIther way, back to our train of thought here. We want visitors to be able to take pictures of everything
else, except for the thing that we have tagged. How do we decide this? One approach is to put guards
around to prevent people from taking any pictures at all. That is certainly effective, however it might not
satisfy psychological acceptability. Alternatively, you might allow people to take any pictures they want,
but then before they leave, you check every single one of their photos to see if your secret device is in
any of the pictures. If so, then you stop them.

In this latter case, how do you identify the object in question? Are you confident that whoever is
checking will be able to identify it? What if the picture is taken from the side? The top? What if it’s a
reflection? Or maybe a shadow?

This illustrates the problem at hand, we can try to enumerate all of the different possibilities, but there 
there has to be a better way of doing it. Theoretically, this known as the non-interference property \cite{Goguen82}. 
It is important to note that the original paper discussed non-interference in the context of classification levels
but it should not be difficult to map our problem into the non-interference model. The property states
that given two things, $A$ and $B$, 

\begin{definition}
$A$ is non-interfering with $B$ if and only if making {\em any} changes to $A$ will {\em not} affect $B$
\end{definition}

In the context of the picture problem, this is the same as asking if
making any changes to the object in question will or will not affect the picture. To see how this 
might work, imagine yourself taking two pictures while I have the opportunity to make any changes 
to the object that you are trying to take a picture of. So there are three steps:

\begin{enumerate}
\item Take picture $p_1$ of the object $o$
\item I make any changes that I want to object $o$
\item Take picture $p_2$ of the same object $o$
\end{enumerate}

If $p_1 \neq p_2$ then I am interfering with you. If they are the same, then I am not interfering
with you because what I did did not affect what you are doing. So if you are taking a selfie in New York
and I am walking around in Amsterdam, then I am not interfering with you. If I happened to be in New York
as well plus I am walking behind you then I am interfering with you since I might photo-bomb your 
selfie. On the other hand, if I was walking in front of you I might not interfere with you, unless
I happened to bump into your selfie-stick or perhaps I am in the process of taking a selfie myself
with the flash on and my flash happens to illuminate you which would make your selfie brighter than
if I wasn't around. This latter example highlights the problem with trying to enumerate all possible
operations. 

Let’s look at a concrete computer example.
As a quick reminder, what we want to know is how the tag for an object should be propagated. In terms of
the basic operations that we are dealing wiht, the objects are actually registers since the CPU 
instructions operate on registers. In particular, there will be input registers as well as output
registers and our goal is to determine whether the output registers should be tagged if the 
input registers are tagged. To put it in non-interference terms, we want to determine if
the input registers interfere with the output registers. As we will see, this will depend, not 
only on whether the registers are tagged, but also what values they contain. 

\begin{definition}
Given an operation, $op$ which take an input $i$ with possible assignments (values) of $v_i$ and 
generates an output $o$ with possible assignments $v_o$, i.e., $v_o = op(v_i)$, 
we say that $i$ is interfering with $o$ if and only if $\exists v_i, v_i' \mid v_o' \neq v_o \land v_i \neq v_i'$
\end{definition}

In the case of a direct copy, propagating tags is
straightforward because the non-interference property is clear. Any changes to the value of the input 
register will directly affect the value of output register. Therefore, if the input is tagged, then 
the copy should also be tagged. The problem gets
complicated when we think about more complex instructions though. Take the multiply instruction as a
quick example. Unlike direct copy that has one input and one output parameter, multiply takes 2 input
parameters and has 1 output parameter.

To support this, we must first expand our definition above to support multiple inputs and outputs.

\begin{definition}
Given an operation, $op$ with takes inputs $I = \{i_1, i_2, ..., i_n\}$ and generates outputs $O = \{o_1, o_2, ..., o_m\}$, 
$i \in I$ is interfering with $o \in O$ if and only if $\exists v_i, v_i' \mid v_o' \neq v_o \land v_i \neq v_i'$. Note 
that all of the other input values $v_j, where i \neq j$ can take on any value -- they are not fixed to whatever the current
values are.
\end{definition}


This means that each input parameter to multiply can have its own tag. So, multiply takes two input
parameters op1 and op2, each of which can have its own tag. For simplicity sake, we will just refer to
these tags as op1\_t and op2\_t. The product is then placed in an output parameter product which also
has a tag called product\_t. Now we must decide how to tag the product. Let’s also assume that the tag is
binary - either the data is tagged or it is not tagged represented by op1\_t = 1 and 0 respectively.
A simple solution might be to say that the product is tagged as long as either of the operands are tagged.
Since we are saying that the tags are binary and can be represented by 1 and 0, we can actually just
calculate the product’s tag using simple bitwise operations: product\_t = op1\_t OR op2\_t. This simple rule
works for most cases and is what someone might come up with as a first try.

However, a case where this doesn’t work is if op1 is 0 and is not tagged (op1\_t = 0). Because anything
multiplied by 0 will result in 0, the product is always guaranteed to be 0. That is, the product has nothing
to do with op2, so it shouldn’t matter whether op2 is tagged or not. Note that this special case doesn’t
apply if the first operand is indeed tagged though.

Given the special case, one might wonder how we can identify all of these cases. We can do this by
applying the non-interference principle which for our purposes state that the output should be tagged if
and only if changing the value of tagged inputs will change the value of the output.

How do we do this? Well normally we might use formal reasoning, however we won’t do that here
because it is way complicated. We will instead use the brute force method and just enumerate all
possibilities, because there really aren’t that many possibilities for this example. There will be a lot in
general though, meaning exhaustive lists like this is not feasible in real life.

To see how things work, we will first have to identify all variables in this problem. In particular, we have
op1, op2, op1\_t, op2\_t, product and product\_t. Given these variables, we want to ask the question of
whether the product should be tagged (i.e. whether product\_t = 1) if op1 is 0 and not tagged. Let’s start
writing down the different combinations where the prime (‘) are used to represent a different value
assignment.

\begin{verbatim}
0 x 0 = product
0 x 1 = product'
0 x 2 = product''
...
0 x op2 = product (the general case)
\end{verbatim}

We should quickly see that it doesn’t matter what we assign op2 to (since we are changing the potential
value of op2, we are basically assuming that op2 is tagged), product = product' = product'' = ... = 0. Since
product never changes, it means that op1 is non-interfering with op2. In this case we didn’t change the
value of op1 to op1’ because op1 is not tagged. What if it was tagged?

If op1 was tagged, then we have to try different assignments as well that includes:

\begin{verbatim}
0 x op2 = product
1 x op2 = product'
2 x op2 = product''
... #all possible values of op1 with a single value of op2
0 x op2' = product'''
1 x op2' = product''''
... #all possible values of op1 with a different value of op2
op1 x op2 = product #all possible values of op1 with all possible values of op2 (the general case)
\end{verbatim}

It should be fairly obvious that product != product’ != product’’ ... This means that the product needs to
be tagged if op1 is tagged. (You can convince yourself that the exception is if op2 is 0 and not tagged).
Either way, we should see that determining whether the output (product) should be tagged depends not
only on 1) the operation (multiply), 2) the tags of the inputs (op1\_t, op2\_t), but also 3) the actual
concrete values of the inputs (op1, op2).

Your assignment asks you to do this for different operations and also with bit-wise tags versus operand
tags. That is, n this example, we have associated tags with entire operands, but this is not necessary. In
fact we can assign tags to entire files down to operands down to individual bits (which is what
memcheck does) -- we call this the {\em tagging granularity.}

